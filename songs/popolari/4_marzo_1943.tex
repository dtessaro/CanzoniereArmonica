%fonte: https://www.harptabs.com/song.php?ID=23986

\beginsong{4 Marzo 1943}[by={Lucio Dalla}]
\musicnote{Armonica: Diatonica}
%\textnote{}

\beginchorus

\[5]\[4] \[ ] \[6]\[5]\[7] \\
\[5]\[4] \[ ] \[6]\[5]\[7] \\
\[5]\[4] \[ ] \[5]\[4] \\
\[2]\[1]\\
\endchorus

\chordsoff
\beginverse
Dice che era un bell'uomo e veniva\\
Veniva dal mare\\
Parlava un'altra lingua\\
Però sapeva amare\\
E quel giorno lui prese a mia madre\\
Sopra un bel prato\\
L'ora più dolce prima d'essere ammazzato\\
\endverse

\beginverse
Così lei restò sola nella stanza \\
La stanza sul porto\\
Con l'unico vestito ogni giorno più corto\\
E benché non sapesse il nome\\
E neppure il paese\\
M'aspettò come un dono d'amore fino dal primo mese\\
\endverse\

\beginverse
Compiva sedici anni quel giorno la mia mamma\\
Le strofe di taverna\\
Le cantò a ninna nanna\\
E stringendomi al petto che sapeva\\
Sapeva di mare\\
Giocava a far la donna con il bimbo da fasciare\\
\endverse

\beginverse
E forse fu per gioco o forse per amore\\
Che mi volle chiamare come nostro Signore\\
Della sua breve vita è il ricord, il ricordo più grosso\\
È tutto in questo nome\\
Che io mi porto addosso\\
\endverse

\beginverse
E ancora adesso che gioco a carte\\
E bevo vino\\
Per la gente del porto mi chiamo Gesù bambino \rep{3}\\

\endverse

\endsong