%fonte:

\beginsong{Bella Ciao}[by={}]
\musicnote{Armonica: Diatonica C}
\textnote{}

\beginverse
\[5]Un\[\asp{6}]a \[\asp{7}]mat\[7]ti\[\asp{6}]na\\
\[5]mi \[\asp{6}]so\[\asp{7}]no \[7]alza\[\asp{6}]to,\\
\[5]O \[\asp{6}]bel\[\asp{7}]la \[7]ciao, \[\asp{7}]bel\[\asp{6}]la \[7]ciao,\\
\[\asp{7}]Bel\[\asp{6}]la \[8]ciao, \[8]ciao, \[8]ciao,\\
\[8]Un\[\asp{8}]a \[8]mat\[\asp{9}]ti\[\asp{9}]na\\
\[\asp{9}]mi \[8]so\[\asp{8}]no \[\asp{9}]alza\[8]to,\\
\[\asp{8}]E \[7]ho \[\asp{7}]trova\[8]to \[7]l'in\[\asp{7}]va\[\asp{6}]sor.\\
\endverse
\chordsoff

\beginverse
O partigiano portami via,\\
O bella ciao, bella ciao,\\
Bella ciao, ciao, ciao,\\
O partigiano portami via,\\
Qui mi sento di moror.\\
\endverse

\beginverse
E so io muoio da partigiano,\\
O bella ciao, bella ciao,\\
Bella ciao, ciao, ciao,\\
E so io muoio da partigiano,\\
Tu mi devi seppellir.\\
\endverse

\beginverse
E seppellire sulla montagna\\
O bella ciao, bella ciao,\\
Bella ciao, ciao, ciao,\\
E seppellire sulla montagna\\
Sott l'ombra di un bel fior.\\
\endverse

\beginverse
Così le genti che passeranno\\
O bella ciao, bella ciao,\\
Bella ciao, ciao, ciao,\\
Casi le genti che passeranno\\
Mi diranno « che bel fior ».\\
\endverse

\beginverse
E questo è il fiore del partigiano\\
O bella ciao, bella ciao,\\
Bella ciao, ciao, ciao,\\
E questo è il fiore del partigiano\\
Morto per la libertà.\\
\endverse

\endsong